\documentclass[a4,12pt]{article}
%\usepackage[margin=1in]{geometry}

%\usepackage[default]{frcursive}
%\usepackage[T1]{fontenc}

\usepackage[T1]{fontenc}
\usepackage{cmbright}

\usepackage[hidelinks]{hyperref}
\usepackage{nameref}

\title{\textbf{MY FAVOURITE POEMS}}
\author{PULKIT JOSHI}
\date{}


\begin{document}
    \maketitle
    \begin{center}
    \newpage
    \tableofcontents
    
        
        % A PRAYER FOR MY DAUGHTER %

        \newpage
        \section*{A Prayer for my Daughter}
        \label{sec:A Prayer for my Daughter}
        \addcontentsline{toc}{section}{\nameref{sec:A Prayer for my Daughter}}
        \subsection*{William Butler Yeats}

        \vspace{15pt}
        Once again the storm is howling, and half hid\\
        Under this cradle-hood and coverlid\\
        My child sleeps on. There is no obstacle\\
        But Gregory's wood and one bare hill\\
        Whereby the haystack- and roof-levelling wind,\\
        Bred on the Atlantic, can be stayed;\\
        And for an hour I have walked and prayed\\
        Because of the great gloom that is in my mind.\\
        
        \vspace{15pt}
        I have walked and prayed for this young child an hour\\
        And heard the sea-wind scream upon the tower,\\
        And under the arches of the bridge, and scream\\
        In the elms above the flooded stream;\\
        Imagining in excited reverie\\
        That the future years had come,\\
        Dancing to a frenzied drum,\\
        Out of the murderous innocence of the sea.\\
       
        \vspace{15pt}
        May she be granted beauty and yet not\\
        Beauty to make a stranger's eye distraught,\\
        Or hers before a looking-glass, for such,\\
        Being made beautiful overmuch,\\
        Consider beauty a sufficient end,\\
        Lose natural kindness and maybe\\
        The heart-revealing intimacy\\
        That chooses right, and never find a friend.\\

        \vspace{15pt}
        Helen being chosen found life flat and dull\\
        And later had much trouble from a fool,\\
        While that great Queen, that rose out of the spray,\\
        Being fatherless could have her way\\
        Yet chose a bandy-leggèd smith for man.\\
        It's certain that fine women eat\\
        A crazy salad with their meat\\
        Whereby the Horn of Plenty is undone.\\

        \vspace{15pt}
        In courtesy I'd have her chiefly learned;\\
        Hearts are not had as a gift but hearts are earned\\
        By those that are not entirely beautiful;\\
        Yet many, that have played the fool\\
        For beauty's very self, has charm made wise,\\
        And many a poor man that has roved,\\
        Loved and thought himself beloved,\\
        From a glad kindness cannot take his eyes.\\

        \vspace{15pt}
        May she become a flourishing hidden tree\\
        That all her thoughts may like the linnet be,\\
        And have no business but dispensing round\\
        Their magnanimities of sound,\\
        Nor but in merriment begin a chase,\\
        Nor but in merriment a quarrel.\\
        O may she live like some green laurel\\
        Rooted in one dear perpetual place.\\

        \vspace{15pt}
        My mind, because the minds that I have loved,\\
        The sort of beauty that I have approved,\\
        Prosper but little, has dried up of late,\\
        Yet knows that to be choked with hate\\
        May well be of all evil chances chief.\\
        If there's no hatred in a mind\\
        Assault and battery of the wind\\
        Can never tear the linnet from the leaf.\\

        \vspace{15pt}
        An intellectual hatred is the worst,\\
        So let her think opinions are accursed.\\
        Have I not seen the loveliest woman born\\
        Out of the mouth of Plenty's horn,\\
        Because of her opinionated mind\\
        Barter that horn and every good\\
        By quiet natures understood\\
        For an old bellows full of angry wind?\\

        \vspace{15pt}
        Considering that, all hatred driven hence,\\
        The soul recovers radical innocence\\
        And learns at last that it is self-delighting,\\
        Self-appeasing, self-affrighting,\\
        And that its own sweet will is Heaven's will;\\
        She can, though every face should scowl\\
        And every windy quarter howl\\
        Or every bellows burst, be happy still.\\

        \vspace{15pt}
        And may her bridegroom bring her to a house\\
        Where all's accustomed, ceremonious;\\
        For arrogance and hatred are the wares\\
        Peddled in the thoroughfares.\\
        How but in custom and in ceremony\\
        Are innocence and beauty born?\\
        Ceremony's a name for the rich horn,\\
        And custom for the spreading laurel tree.\\
        

        % BECAUSE I COULD NOT STOP FOR DEATH %
        \newpage
        \section*{Because I could not stop for Death}
        \label{sec:Because I could not stop for Death}
        \addcontentsline{toc}{section}{\nameref{sec:Because I could not stop for Death}}
        \subsection*{Emily Dickinson}

        \vspace{15pt}
        Because I could not stop for Death –\\
        He kindly stopped for me –\\
        The Carriage held but just Ourselves –\\
        And Immortality.\\

        \vspace{15pt}
        We slowly drove – He knew no haste\\
        And I had put away\\
        My labor and my leisure too,\\
        For His Civility –\\

        \vspace{15pt}
        We passed the School, where Children strove\\
        At Recess – in the Ring –\\
        We passed the Fields of Gazing Grain –\\
        We passed the Setting Sun –\\

        \vspace{15pt}
        Or rather – He passed us –\\
        The Dews drew quivering and chill –\\
        For only Gossamer, my Gown –\\
        My Tippet – only Tulle –\\

        \vspace{15pt}
        We paused before a House that seemed\\
        A Swelling of the Ground –\\
        The Roof was scarcely visible –\\
        The Cornice – in the Ground –\\

        \vspace{15pt}
        Since then – 'tis Centuries – and yet\\
        Feels shorter than the Day\\
        I first surmised the Horses' Heads\\
        Were toward Eternity –\\


        % IF %
        \newpage
        \section*{If}
        \label{sec:If}
        \addcontentsline{toc}{section}{\nameref{sec:If}}
        \subsection*{Rudyard Kipling}

        \vspace{15pt}
        If you can keep your head when all about you\\
        Are losing theirs and blaming it on you;\\
        If you can trust yourself when all men doubt you,\\
        But make allowance for their doubting too;\\
        If you can wait and not be tired by waiting,\\
        Or, being lied about, don’t deal in lies,\\
        Or, being hated, don’t give way to hating,\\
        And yet don’t look too good, nor talk too wise;\\

        \vspace{15pt}
        If you can dream—and not make dreams your master;\\
        If you can think—and not make thoughts your aim;\\
        If you can meet with triumph and disaster\\
        And treat those two impostors just the same;\\
        If you can bear to hear the truth you’ve spoken\\
        Twisted by knaves to make a trap for fools,\\
        Or watch the things you gave your life to broken,\\
        And stoop and build ’em up with wornout tools;\\

        \vspace{15pt}
        If you can make one heap of all your winnings\\
        And risk it on one turn of pitch-and-toss,\\
        And lose, and start again at your beginnings\\
        And never breathe a word about your loss;\\
        If you can force your heart and nerve and sinew\\
        To serve your turn long after they are gone,\\
        And so hold on when there is nothing in you\\
        Except the Will which says to them: “Hold on”;\\

        \vspace{15pt}
        If you can talk with crowds and keep your virtue,\\
        Or walk with kings—nor lose the common touch;\\
        If neither foes nor loving friends can hurt you;\\
        If all men count with you, but none too much;\\
        If you can fill the unforgiving minute\\
        With sixty seconds’ worth of distance run—\\
        Yours is the Earth and everything that’s in it,\\
        And—which is more—you’ll be a Man, my son!\\

 

    \end{center}
\end{document}
